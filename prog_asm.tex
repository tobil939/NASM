\documentclass[a4paper,12pt,twoside]{article}
\usepackage[ngerman]{babel}
\usepackage[utf8]{inputenc}
\usepackage[T1]{fontenc}
\usepackage{geometry}
\usepackage{fancyhdr}
\usepackage[hidelinks]{hyperref}
\usepackage{listings}
\usepackage{xcolor}
\usepackage{ulem}

\geometry{
  a4paper, 
  left=3cm,
  right=3cm,
  top=2cm,
  bottom=2cm
}

% Farben für Code
\definecolor{codebg}{RGB}{245,245,245}
\definecolor{keyword}{RGB}{0,0,180}
\definecolor{comment}{RGB}{0,128,0}
\definecolor{string}{RGB}{163,21,21}
\definecolor{directive}{RGB}{128,0,128}

% NASM Syntax Highlighting
\lstdefinelanguage{NASM}{
  morekeywords={mov, add, sub, mul, div, jmp, jz, jnz, call, ret, push, pop, int, xor, and, or, cmp, lea},
  morekeywords=[2]{section, global, extern, db, dw, dd, dq, resb, resw, resd, resq},
  morekeywords=[3]{eax, ebx, ecx, edx, esi, edi, esp, ebp, ax, bx, cx, dx, al, ah, bl, bh, cl, ch, dl, dh},
  sensitive=true,
  morecomment=[l]{;}, % Single-line comments
  morestring=[b]",    % Strings in double quotes
  keywordstyle=\color{keyword}\bfseries,
  keywordstyle=[2]\color{directive}\bfseries,
  keywordstyle=[3]\color{blue}\bfseries,
  commentstyle=\color{comment},
  stringstyle=\color{string}
}

\lstset{
  language=NASM,
  backgroundcolor=\color{codebg},
  basicstyle=\ttfamily\footnotesize,
  numbers=left,
  numberstyle=\tiny,
  breaklines=true,
  frame=single,
  showstringspaces=false,
  tabsize=4
}

% Titelseite
\title{Ich lerne Assembly x86 NASM}
\author{Tobi}
\date{\today}

% Kopf und Fußzeilen
\pagestyle{fancy}
\fancyhf{} % Kopf- und Fußzeilen leeren
\fancyfoot[C]{\thepage} % Seitenzahl zentriert in der Fußzeile
\fancyhead[LE, RO]{\title}
\fancyhead[LO, RE]{\author}


\begin{document}

% Titelseite
\maketitle
\thispagestyle{empty} % Keine Kopf-/Fußzeile auf der Titelseite
\newpage

% Inhaltsverzeichnis
\tableofcontents
\newpage

% Beispielkapitel
\section{Ausführen}
\subsection{Linux}
\verb|nasm -f elf32 hello.asm -o hello.o| 
\verb|ld -m elf_i386 -o hello hello.o|
\verb|./hello|
\section{Debugger}
\subsection{Linux}
\verb|nasm -f elf32 -g -F dwarf hello.asm -o hello.o|
\verb|ld -m elf_i386 -o hello hello.o|
\verb|gdb ./hello|
\begin{tabbing}
  \hspace{2mm} \= \hspace{50mm} \= \kill
  \> \verb|break _start| \> setzt einen Breakpoint \\ 
  \> \verb|run| \> führt das Programm im Debugger aus \\ 
  \> \verb|stepi| \> führt einen einzelne Assembler-Instruktion aus \\ 
  \> \verb|nexti| \> nächste Instruktion ohne Eintritt in Funktionen \\ 
  \> \verb|info registers| \> zeigt den Status aller Register an \\ 
  \> \verb|x/10b $esp| \> zeigt den Speicherinhalt ab Stackpointer an \\ 
  \> \> Stackpointer \verb|esp|; Befehl zum Untersuchen \verb|x| \\ 
  \> \> zeigt 10 Bytes an \verb|10b| \\ 
  \> \verb|continue| \> setzt das Programm fort bis zum nächsten Breakpoint \\ 
  \> \verb|quit| \> beendet den Debugger \\
\end{tabbing}
\section{Struktur}
\subsection{Hauptprogramm}
\begin{center}
  \begin{minipage}{1.0\textwidth}
    \begin{lstlisting}[language=NASM]
section .text     ; definiert einen Teil des Programms, Name ist .text
global _start     ; definiert das Programm als global

_start:
  ; Hauptprogramm beginnt hier 
  call sub_program   ; Aufruf des Unterprogramms 
    \end{lstlisting}
  \end{minipage}
\end{center}
\subsection{Unterprogramm}
\begin{center}
  \begin{minipage}{1.0\textwidth}
    \begin{lstlisting}[language=NASM]
sub_programm:
  ; Unterprogramm beginnt hier
  ret   ; Rueckkerh ins Hauptprogramm
    \end{lstlisting}
  \end{minipage}
\end{center}
\section{Syntax}
\begin{tabbing}
  \hspace{2mm} \= \hspace{50mm} \= \kill 
  \> \verb|;Kommentar| \> erzeugt ein Kommentar \\ 
  \> \verb|[ ]| \> Runde Klammern werden kaum verwendet \\ 
  \> \verb|{ }| \> Geschweifte Klammern werden kaum verwendet \\ 
  \> \verb|,| \> Trennzeichen in Befehlen \\
  \> \verb|.| \> Direktiven, definiert Anweisungen oder Daten \\ 
  \> \verb|.bunga| \> lokales Label, nur inerhalb eines Prozesses \\ 
  \> \verb|_bunga| \> Label, vorallem für C-kompatible Codes \\ 
  \> \verb|' '| \> String \\
\end{tabbing}
\section{Ein und Ausgabe}
\subsection{Ausgabe}
\section{Register}
\subsection{Accumulator}
Wird oft als allgemeines Zweckregister verwendet, hat historisch aber eine \\ 
besondere Bedeutung. \\ 
Definition: eax, ax oder al \\ 
Bei Multiplikation und Division wird der Accumulator als erster Operant gerundet. 
Wird auch häufig für I/O verwendet \\ 
\subsection{Base Pointer}
Basisregister für Speicheradressierungen oder Index-Register. \\ 
Definition ebx, bx oder bl \\
\subsection{64-Bit-Register}
\begin{tabbing}
  \hspace{2mm} \= \hspace{50mm} \= \kill 
  \> rax \> Accumulator \\ 
  \> rbx \> Base Pointer \\ 
  \> rcx \> Counter \\ 
  \> rdx \> Data \\ 
  \> rsi \> Source Index \\ 
  \> rdi \> Destination Index \\ 
  \> rbp \> Base Pointer fuer den Stack Frame \\ 
  \> rsp \> Stack Pointer \\
\end{tabbing}
\subsection{32-Bit-Register}
\begin{tabbing}
  \hspace{2mm} \= \hspace{50mm} \= \kill 
  \> eax \> Accumulator \\ 
  \> ebx \> Base Pointer \\ 
  \> ecx \> Counter \\ 
  \> edx \> Data \\ 
  \> esi \> Source Index \\ 
  \> edi \> Destination Index \\ 
  \> ebp \> Base Pointer fuer den Stack Frame \\ 
  \> esp \> Stack Pointer \\
\end{tabbing}
\subsection{16-Bit-Register}
\begin{tabbing}
  \hspace{2mm} \= \hspace{50mm} \= \kill 
  \> ax \> Accumulator \\ 
  \> bx \> Base Pointer \\ 
  \> cx \> Counter \\ 
  \> dx \> Data \\ 
\end{tabbing}
\subsection{8-Bit-Register}
\begin{tabbing}
  \hspace{2mm} \= \hspace{50mm} \= \kill 
  \> ah \> oberes Byte von ax \\ 
  \> al \> unteres Byte von ax \\ 
  \> bh \> oberes Byte von bx \\ 
  \> bl \> unteres Byte von bx \\ 
  \> ch \> oberes Byte von cx \\ 
  \> cl \> unteres Byte von cx \\ 
  \> dh \> oberes Byte von dx \\ 
  \> dl \> unteres Byte von dx \\ 
\end{tabbing}
\section{Variablen}
\verb|section .data     ; Daten die einen Anfangswert haben| 
\verb|section .bss      ; erstellt Platzhalter fuer Variablen| 
\section{Datentypen}
\begin{tabbing}
  \hspace{2mm} \= \hspace{50mm} \= \kill 
  \> db \> define byte, 8bit \\ 
  \> dw \> define word, 16bit \\ 
  \> dd \> define double word, 32bit \\ 
  \> dq \> define quadword, 64bit \\ 
  \> dt \> define ten bytes, 80bit \\ 
  \> ddq \> define double quadword, 128bit \\ 
  \> do \> define octword 16Bytes \\ 
  \> dy \> define yword 32Bytes \\ 
  \> dz \> define zword 64Bytes \\ 
  \> times \> Array \\
  \> array\_bunga times 10 db 0 \> definiert ein Array mit 10 byte \\  
  \> bunga db 'Hallo' 0 \> defniert eine 8bit String \\
  \> resb \> reserve byte \\ 
  \> resw \> reserve word \\ 
  \> resd \> reserve double word \\ 
  \> resq \> reserve quadword \\ 
  \> rest \> reserve ten bytes \\ 
  \> resdq \> reserve double quadword \\ 
  \> reso \> reserve octword \\ 
  \> resy \> reserve yword \\ 
  \> resz \> reserve zword \\ 
\end{tabbing}
\textbf{Struktur}
\begin{center}
\begin{minipage}{1.0\textwidth}
  \begin{lstlisting}[language=NASM]
person:
  name    db 'Horst', 0 
  age     dw 30 
  height  dd 175
  \end{lstlisting}
\end{minipage}
\end{center}
\textbf{Fließkommazahlen}
\begin{center}
\begin{minipage}{1.0\textwidth}
  \begin{lstlisting}[language=NASM]
; Kommazahlen muessen hexadezimal dargestellt werden 
; IEEE-754
; 1.0 in Single-Precision
single_float dd 0x3f800000 

; 1.0 in Double-Precision 
double_float dq 0x3ff0000000000000
  \end{lstlisting}
\end{minipage}
\end{center}
\section{Datentransfer}
\begin{tabbing}
  \= \hspace{2mm} \= \hspace{50mm} \= \kill 
  \> mov \> kopiert von Quelle nach Ziel \\ 
  \> xchg \> tauscht zwei Werte \\ 
  \> push \> speichert Wert auf Stack \\ 
  \> pop \> holt Wert von Stack \\ 
  \> enter \> erstellt einen Stackframe \\ 
  \> leave \> beendet das Stackframe \\
  \> pusha \> speichert alle allgemeinen Register auf dem Stack \\ 
  \> popa \> holt alle allgemeinen Register von dem Stack \\ 
  \> pushf \> speichert Flags auf dem Stack \\ 
  \> popf \> holt Flags vom Stack \\ 
  \> lahf \> holt die unteren 8bit von FLAGS in AH \\ 
  \> sahf \> speichert AH zurueck in FLAGS \\ 
  \> in \> liest einen I/O Port \\ 
  \> out \> schreibt auf einen I/O Port \\ 
  \> lea \> laedt die Adresse eines Speichers \\ 
  \> lds \> laedt Registerpaar mit Zeiger und Segment \\ 
  \> les \> laedt Registerpaar mit Zeiger und Segment \\ 
  \> lfs \> laedt Registerpaar mit Zeiger und Segment \\ 
  \> lgs \> laedt Registerpaar mit Zeiger und Segment \\ 
  \> lss \> laedt Registerpaar mit Zeiger und Segment \\ 
  \> movX \> X kann s, b, w oder d sein \\ 
  \> \> kopiert eine Zeichenfolge \\
  \> cmpX \> X kann s, b, w oder d sein \\
  \> \> Vergleicht Zeichenfolge \\
  \> scaX \> X kann s, b, w oder d sein \\ 
  \> \> sucht nach Zeichenfolge \\ 
  \> stoX \> X kann s, b, w oder d sein \\ 
  \> \> speichert Zeichenfolge \\
\end{tabbing}
\section{Schleifen}
\subsection{Sprünge}
\subsubsection{Unbedingte Sprünge}
\hspace{2mm} jmp \hspace{50mm} springt zu einem Label \\
\subsubsection{Bedingte Sprünge}
Verwendet die Flags:
\hspace{2mm} CF \hspace{50mm} Carry Flag (für unsigned Vergleiche) \\ 
\hspace{2mm} ZF \hspace{50mm} Zero Flag \\ 
\hspace{2mm} SF \hspace{50mm} Sign Flag (für signed Vergleiche) \\ 
\hspace{2mm} OF \hspace{50mm} Overflow Flag (für signed Vergleiche) \\
\hspace{2mm} CX \hspace{50mm} Register mit 16bit \\ 
\hspace{2mm} ECX \hspace{50mm} Register mit 32bit \\ 
\hspace{2mm} RCX \hspace{50mm} Register mit 64bit \\
\begin{tabbing}
  \hspace{2mm} \= \hspace{50mm} \= \kill 
  \> je \> jump when equal (ZF=1, nach \texttt{cmp}) \\ 
  \> jz \> jump when zero (ZF=1, nach Berechnung, synonym zu je) \\ 
  \> jne \> jump when not equal (ZF=0) \\ 
  \> jnz \> jump when not zero (ZF=0, synonym zu jne) \\ 
  \> jg \> jump when greater (signed, ZF=0 \& SF=OF) \\ 
  \> jnle \> jump when not less or equal (signed, ZF=0 \& SF=OF, synonym zu jg) \\ 
  \> jge \> jump when greater or equal (signed, SF=OF) \\ 
  \> jnl \> jump when not less (signed, SF=OF, synonym zu jge) \\ 
  \> jl \> jump when less (signed, SF$\neq$OF) \\ 
  \> jnge \> jump when not greater or equal (signed, SF$\neq$OF, synonym zu jl) \\ 
  \> jle \> jump when less or equal (signed, ZF=1 OR SF$\neq$OF) \\ 
  \> jng \> jump when not greater (signed, ZF=1 OR SF$\neq$OF, synonym zu jle) \\ 
  \> ja \> jump if above (CF=0 und ZF=0, unsigned greater) \\ 
  \> jae \> jump if above or equal (CF=0, unsigned greater or equal) \\ 
  \> jb \> jump if below (CF=1, unsigned less) \\ 
  \> jbe \> jump if below or equal (CF=1 or ZF=1, unsigned less or equal) \\
  \> jcxz \> jump if CX is Zero (CX = 0) \\
  \> jecxz \> jump if ECX is Zero (ECX = 0) \\
  \> jrcxz \> jump if RCX is Zero (RCX = 0) \\
\end{tabbing}
\begin{center}
  \begin{minipage}{1.0\textwidth}
    \begin{lstlisting}[language=NASM]
; Sprung wenn 10 > 5 (unsigned)
mov eax, 10
mov ebx, 5
cmp eax, ebx   ; 10 - 5 entspricht CF=0, ZF=0
ja groesser    ; Springt, weil 10 > 5

mov eax, 1     ; Wird uebersprungen

groesser:
mov eax, 2     ; eax = 2 (weil gesprungen)
    \end{lstlisting}
  \end{minipage}
\end{center}
\subsection{loop}
\begin{center}
  \begin{minipage}{1.0\textwidth}
    \begin{lstlisting}[language=NASM]
section .text
global _start

_start:
    mov ecx, 5  ; Initialisiere ecx mit 5 Wiederholungen
loop_start:
    ; Schleife
    ; Hier koennte man z.B. etwas ausgeben oder eine Operation durchfuehren
    loop loop_start  ; Springe zurueck zu loop_start, solange ecx > 0

    ; Schleifenende
    ; Programm beenden
    mov eax, 1
    xor ebx, ebx
    int 0x80
    \end{lstlisting}
  \end{minipage}
\end{center}
\hspace{2mm} loopz \hspace{50mm} loop while zero \\ 
\hspace{2mm} loope \hspace{50mm} loop while equal \\ 
\hspace{2mm} loopnz \hspace{50mm} loop while not zero \\ 
\hspace{2mm} loopne \hspace{50mm} loop while not equal \\ 
\subsection{While}
\begin{center}
  \begin{minipage}{1.0\textwidth}
    \begin{lstlisting}[language=NASM]
section .text
global _start

_start:
    mov ecx, 0  ; Initialisiere ecx als Zaehler

while_loop:
    ; Bedingung ueberpruefen
    cmp ecx, 5  ; Vergleiche ecx mit 5
    jge end_while  ; Springe zu end_while, wenn ecx >= 5

    ; Schleifenkoerper
    inc ecx  ; Erhoehe ecx um 1
    ; Hier koennten weitere Operationen ausgefuehrt werden
    jmp while_loop  ; Zueck zum Bedingungscheck

end_while:
    ; Schleifenende, Programm fortsetzen

    ; Programm beenden
    mov eax, 1
    xor ebx, ebx
    int 0x80
    \end{lstlisting}
  \end{minipage}
\end{center}
\subsection{Do-While}
\begin{center}
  \begin{minipage}{1.0\textwidth}
    \begin{lstlisting}[language=NASM]
section .text
global _start

_start:
    mov ecx, 0  ; Initialisiere ecx als Zaehler

do_while_loop:
    ; Schleifenkoerper
    inc ecx  ; Erhoehe ecx um 1
    ; Hier koennten weitere Operationen ausgefuehrt werden

    ; Bedingung ueberpruefen (am Ende der Schleife)
    cmp ecx, 5
    jl do_while_loop  ; Springe zurueck, wenn ecx < 5

    ; Schleifenende
    ; Programm fortsetzen

    ; Programm beenden
    mov eax, 1
    xor ebx, ebx
    int 0x80
    \end{lstlisting}
  \end{minipage}
\end{center}
\subsection{For}
\begin{center}
  \begin{minipage}{1.0\textwidth}
    \begin{lstlisting}[language=NASM]
section .text
global _start

_start:
    mov ecx, 0  ; Startwert
    mov edx, 5  ; Endwert

for_loop:
    ; Schleifenkoerper
    ; Hier koennte man mit ecx als Index arbeiten
    inc ecx  ; Erhoehe ecx (Index) um 1

    ; Bedingung ueberpruefen
    cmp ecx, edx  ; Vergleiche ecx mit Endwert
    jle for_loop  ; Springe zurueck, wenn ecx <= edx

    ; Schleifenende
    ; Programm fortsetzen

    ; Programm beenden
    mov eax, 1
    xor ebx, ebx
    int 0x80
    \end{lstlisting}
  \end{minipage}
\end{center}
\section{Funktionen}
\subsection{call}
Unterprogramme haben ein label, das label wird dann mit call aufgerufen. \\ 
Sie enden meist mit dem ret Befehl. 
\begin{center}
  \begin{minipage}{1.0\textwidth}
    \begin{lstlisting}[language=NASM]
section .text
global _start

_start:
    ; Hauptprogramm
    call my_function  ; Rufe das Unterprogramm auf
    ; Weitere Logik nach dem Aufruf

    ; Beende das Programm
    mov eax, 1
    xor ebx, ebx
    int 0x80

my_function:
    ; Code des Unterprogramms
    ; Zum Beispiel, setze ein Register
    mov eax, 42
    ; Rueckkehr ins Hauptprogramm
    ret
    \end{lstlisting}
  \end{minipage}
\end{center}
Es können auch Parameter und Rückgabewerte definiert werden 
\begin{center}
  \begin{minipage}{1.0\textwidth}
    \begin{lstlisting}[language=NASM]
; Beispiel fuer Parameteruebergabe ueber Register
mov eax, 10
mov ebx, 20
call add_two_numbers
; Ergebnis ist nun in eax

add_two_numbers:
    add eax, ebx
    ret
    \end{lstlisting}
  \end{minipage}
\end{center}
\subsection{Interrupt}
\subsubsection{Sustemaufrufe Linux}
\hspace{2mm} int 0x80 \hspace{50mm} 32bit Systemaufruf \\ 
\> syscall \> 64bit Systemaufruf \\
\subsubsection{Software-Interrupt}
\begin{center}
  \begin{minipage}{1.0\textwidth}
    \begin{lstlisting}[language=NASM]

      
    \end{lstlisting}
    
  \end{minipage}
  
\end{center}
\section{Multiprozess}
\section{Mathematik}
\begin{tabbing}
  \hspace{2mm} \= \hspace{50mm} \= \kill 
  \> add \> $eax = eax + ebx$ \\ 
  \> sub \> $eax = eax - ebx$ \\ 
  \> mul bl \> Ergebnis in AX 8bit \\ 
  \> mul bx \> Ergebnis in DX:AX 16bit \\ 
  \> mul ebx \> Ergebnis in EDX:EAX \\ 
  \> imul bl \> Ergebnis in AX 8bit mit Vorzeichenbehandlung \\ 
  \> imul bx \> Ergebnis in DX:AX 16bit mit Vorzeichenbehandlung \\ 
  \> imul ebx \> Ergebnis in EDX:EAX 32bit mit Vorzeichenbehandlung \\
  \> div bl \> Quotient in AL, Rest in AH 8bit \\ 
  \> div bx \> Quotient in AX, Rest in DX 16bit \\ 
  \> div ebx \> Quotient in EAX, Rest in EDX 32bit \\
  \> idiv bl \> Quotient in AL, Rest in AH 8bit mit Vorzeichenbehandlung \\ 
  \> idiv bx \> Quotient in AX, Rest in DX 16bit mit Vorzeichenbehandlung \\ 
  \> idiv ebx \> Quotient in EAX, Rest in EDX 32bit mit Vorzeichenbehandlung \\
  \> inc eax \> Inkrementiert um 1 \\ 
  \> dec eax \> Dekrementiert um 1 \\
  \> neg eax \> negiert das Vorzeichen \\ 
  \> adc \> addiert mit Carry \\ 
  \> sbb \> subtrahiert mit Borrow \\
\end{tabbing}
\subsection{Arbeiten mit Fließkommazahlen}
\subsubsection{x87 FPU}
\textbf{x87 Floating-Point Unit}
\begin{tabbing}
  \hspace{2mm} \= \hspace{50mm} \= \kill
  \> fld \> laedt einen Wert auf den FPU-Stack \\ 
  \> fld1 \> laedt die Konstante 1.0 \\ 
  \> fldz \> laedt die Konstaten 0.0 \\ 
  \> fldpi \> laedt die Konstante $\pi$ \\ 
  \> fst \> speichert den Wert von ST(0) ohne Stack Modifikation \\ 
  \> fstp \> speichert den Wert von ST(0) und poppt ihn vom Stack \\
  \> fadd \> $ST(0) = ST(0) + ST(1)$ \\
  \> faddp \> $ST(1) = ST(0) + ST(1)$ poppt ST(0) \\ 
  \> fsub \> $ST(0) = ST(0) - ST(1)$ \\ 
  \> fsubp \> $ST(1) = ST(1) - ST(0)$ poppt ST(0) \\ 
  \> fsubr \> $ST(0) = ST(1) - ST(0)$ \\ 
  \> fsubrp \> $ST(1) = ST(0) - ST(1)$ poppt ST(0) \\ 
  \> fmul \> $ST(1) = ST(0) * ST(1)$ \\ 
  \> fmulp \>$ST(1) = ST(0) * ST(1)$ poppt ST(0) \\
  \> fdiv \> $ST(0) = ST(0) / ST(1)$ \\
  \> fdivp \> $ST(1) = ST(1) / ST(0)$ poppt ST(0) \\ 
  \> fdivr \> $ST(0) = ST(1) / ST(0)$ \\ 
  \> fdivrp \> $ST(1) = ST(0) / ST(1)$ poppt ST(0)\\ 
  \> fsqrt \> Quadratwurzel von ST(0) \\ 
  \> fabs \> setzt ST(0) auf den absolut Wert \\ 
  \> fchs \> wechselst das Vorzeichen von ST(0) \\ 
  \> frndint \> rundet ST(0) auf Ganzzahl \\ 
  \> fcom \> vergleicht ST(0) mit ST(1) \\ 
  \> fcomp \> vergleicht ST(0) mit ST(1) und poppt \\ 
  \> fcompp \> vergleicht ST(0) mit ST(1) und poppt 2x \\
  \> fild \> laedt einen Integer auf den FPU-Stack \\ 
  \> fistp \> Speichert ST(0) als Integer und poppt ST(0) \\ 
  \> \> es wurd normal gerundet \\ 
  \> fldcw \> setzt den Rundungsmodus auf abrunden \\ 
  \> frndint \> schneidet die Zahlen nach dem Komma ab \\ 
  \> fxch \> tauscht ST(0) mit ST(n) \\ 
  \> fstsw \> speichert den Status-Wort der FPU im Register oder Speicher \\ 
  \> fincstp \> erhoeht den Stack-Pointer \\ 
  \> fdecstp \> verringert den Stack-Pointer \\ 
\end{tabbing}
\subsubsection{SEE}
\textbf{Streaming SIMD Extensions}
\begin{tabbing}
  \hspace{2mm} \= \hspace{50mm} \= \kill
  \> movss xmm0, [Variablen] \>  laedt die Variable in ein 32bit Register \\ 
  \> movsd xmm0, [Variablen] \> laedt die Variable in ein 64bit Register \\ 
  \> addss \> addiert scalar 32bit \\ 
  \> addps \> addiert packed 32bit \\ 
  \> subss \> subtrahiert scalar 32bit \\ 
  \> subps \> subtrahiert packed 32bit \\ 
  \> mulss \> multipliziert scalar 32bit \\ 
  \> mulps \> multipliziert packed 32bit \\ 
  \> divss \> dividiert scalar 32bit \\ 
  \> divps \> dividiert packed 32bit \\ 
  \> sqrtss \> Quadratwurzel scalar 32bit \\ 
  \> sqrtps \> Quadratwurzel packed 32bit \\ 
  \> minss \> minimum scalar 32bit \\ 
  \> minps \> minimum packed 32bit \\ 
  \> maxss \> maximum scalar 32bit \\ 
  \> maxps \> maximum packed 32bit \\ 
  \> addsd \> addiert scalar 64bit \\ 
  \> addpd \> addiert packed 64bit \\ 
  \> subsd \> subtrahiert scalar 64bit \\ 
  \> subpd \> subtrahiert packed 64bit \\ 
  \> mulsd \> multipliziert scalar 64bit \\ 
  \> mulpd \> multipliziert packed 64bit \\ 
  \> divsd \> dividiert scalar 64bit \\ 
  \> divpd \> dividiert packed 64bit \\ 
  \> sqrtsd \> Quadratwurzel scalar 64bit \\ 
  \> sqrtpd \> Quadratwurzel packed 64bit \\ 
  \> minsd \> minimum scalar 64bit \\ 
  \> minpd \> minimum packed 64bit \\ 
  \> maxsd \> maximum scalar 64bit \\ 
  \> maxpd \> maximum packed 64bit \\ 
  \> scalar \> einzelnes Element eines Registers \\ 
  \> packed \> mehrere Elemente eines Register \\ 
\end{tabbing}
\subsubsection{AVX}
\textbf{Advanced Vector Extensions}
\begin{tabbing}
  \hspace{2mm} \= \hspace{50mm} \= \kill
  \> vmovss xmm0, [Variablen] \>  laedt die Variable in ein 32bit Register \\ 
  \> vmovsd xmm0, [Variablen] \> laedt die Variable in ein 64bit Register \\ 
  \> vaddss \> addiert scalar 32bit \\ 
  \> vaddps \> addiert packed 32bit \\ 
  \> vsubss \> subtrahiert scalar 32bit \\ 
  \> vsubps \> subtrahiert packed 32bit \\ 
  \> vmulss \> multipliziert scalar 32bit \\ 
  \> vmulps \> multipliziert packed 32bit \\ 
  \> vdivss \> dividiert scalar 32bit \\ 
  \> vdivps \> dividiert packed 32bit \\ 
  \> vsqrtss \> Quadratwurzel scalar 32bit \\ 
  \> vsqrtps \> Quadratwurzel packed 32bit \\ 
  \> vminss \> minimum scalar 32bit \\ 
  \> vminps \> minimum packed 32bit \\ 
  \> vmaxss \> maximum scalar 32bit \\ 
  \> vmaxps \> maximum packed 32bit \\ 
  \> vaddsd \> addiert scalar 64bit \\ 
  \> vaddpd \> addiert packed 64bit \\ 
  \> vsubsd \> subtrahiert scalar 64bit \\ 
  \> vsubpd \> subtrahiert packed 64bit \\ 
  \> vmulsd \> multipliziert scalar 64bit \\ 
  \> vmulpd \> multipliziert packed 64bit \\ 
  \> vdivsd \> dividiert scalar 64bit \\ 
  \> vdivpd \> dividiert packed 64bit \\ 
  \> vsqrtsd \> Quadratwurzel scalar 64bit \\ 
  \> vsqrtpd \> Quadratwurzel packed 64bit \\ 
  \> vminsd \> minimum scalar 64bit \\ 
  \> vminpd \> minimum packed 64bit \\ 
  \> vmaxsd \> maximum scalar 64bit \\ 
  \> vmaxpd \> maximum packed 64bit \\ 
  \> scalar \> einzelnes Element eines Registers \\ 
  \> packed \> mehrere Elemente eines Register \\ 
\end{tabbing}
\textbf{Beispiel}
\begin{center}
\begin{minipage}{1.0\textwidth}
  \begin{lstlisting}[language=NASM]
section .data
  num1 dd 0x401E6666      ; 2.45 als Hex
  num2 dd 0x42355C29      ; 45.34 als Hex

section .text
  global _start

_start:
  movss xmm0, [num1]      ; laedt num1 ins Register xmm0
  movss xmm1, [num2]      ; laedt num2 ins Register xmm1

  addss xmm0, xmm1        ; num1 = num1 + num2

  movss [result], xmm0    ; speichert Ergebnis in xmm0 

  ;Beendet das Programm 
  mov eax, 1              ; SYS_exit 
  xor ebx, ebx            ; exit code 0 
  int 0x80               ; System call fuer Exit
  \end{lstlisting}
\end{minipage}
\end{center}
\section{Logische und Bit-Manipulationsbefehle}
\begin{tabbing}
  \hspace{2mm} \= \hspace{50mm} \= \kill 
  \> and \> Bitweise UND \\ 
  \> or \> Bitweise ODER \\ 
  \> xor \> Bitweise Exklusiv-ODER \\ 
  \> not \> Bitweise Negation \\ 
  \> shl \> links schieben, ohne Vorzeichen  \\ 
  \> sal \> links schieben, mit Vorzeichen \\
  \> shr \> rechts schieben, ohne Vorzeichen \\ 
  \> sar \> rechts schieben, mit Vorzeichen \\
  \> rol \> rotiert nach links \\ 
  \> ror \> rotiert nach rechts \\ 
  \> rcl \> rotiert nach links mit Carry \\ 
  \> rcr \> rotiert nach rechts mit Carry \\ 
  \> bt \> testet ein Bit \\ 
  \> bts \> setzt ein Bit \\ 
  \> btr \> loescht ein Bit \\ 
  \> btc \> komplementiert ein Bit \\ 
\end{tabbing}
\section{Pakete}
\section{Hello World!}
\begin{center}
  \begin{minipage}{1.0\textwidth}
    \begin{lstlisting}[language=NASM]
section .data
message db 'Hello, World!', 0xa ; '0xa' ist der ASCII-Code fuer Zeilenumbruch
length equ $ - message           ; Laenge der Nachricht

section .text
    global _start

_start:
    ; Systemaufruf fuer write (sys_write)
    mov eax, 4          ; Systemaufrufnummer 4 fuer write
    mov ebx, 1          ; Dateideskriptor 1 ist stdout
    mov ecx, message    ; Adresse der Nachricht
    mov edx, length     ; Laenge der Nachricht
    int 0x80            ; Unterbrechung, um den Systemaufruf auszufuehren

    ; Systemaufruf fuer exit (sys_exit)
    mov eax, 1          ; Systemaufrufnummer 1 fuer exit
    xor ebx, ebx        ; Rueckgabewert 0 (ohne Fehler)
    int 0x80            ; Unterbrechung, um den Systemaufruf auszufuehren	  
    \end{lstlisting}
  \end{minipage}
\end{center}
\section{Installation}
\subsection{Kompiler}
\verb|sudo pacman -S nasm|
\subsection{Debugger}
\verb|sudo pacman -S gdb|
\end{document}

\begin{center}
  \begin{minipage}{1.0\textwidth}
    \begin{lstlisting}[language=Assembler]
section .data
message db 'Hello, World!', 0xa ; '0xa' ist der ASCII-Code für Zeilenumbruch
length equ $ - message           ; Länge der Nachricht

section .text
    global _start

_start:
    ; Systemaufruf für write (sys_write)
    mov eax, 4          ; Systemaufrufnummer 4 für write
    mov ebx, 1          ; Dateideskriptor 1 ist stdout
    mov ecx, message    ; Adresse der Nachricht
    mov edx, length     ; Länge der Nachricht
    int 0x80            ; Unterbrechung, um den Systemaufruf auszuführen

    ; Systemaufruf für exit (sys_exit)
    mov eax, 1          ; Systemaufrufnummer 1 für exit
    xor ebx, ebx        ; Rückgabewert 0 (ohne Fehler)
    int 0x80            ; Unterbrechung, um den Systemaufruf auszuführen	  \end{lstlisting}
  \end{minipage}
\end{center}

\begin{tabbing}
  \hspace{5mm} \= \hspace{50mm} \= \kill
  \> \>
\end{tabbing}

\begin{itemize}
  \item
  
\end{itemize}

\begin{enumerate}
  \item
  
\end{enumerate}
